\documentclass[a4paper,10pt]{article}
\usepackage[utf8]{inputenc}

%opening
\title{2-Approximation des Pricecollecting-Steintree-Problems}
\author{}

\begin{document}

\maketitle

\begin{abstract}

\end{abstract}

\section{Motivation}

\section{Einleitung}

\section{Verwendeter Algorithmus}

\section{Implementation}
\subsection{Repräsentation des Graphen}
Zur Repräsentation des Graphen werden zwei Datenstrukturen verwendet.\\
\\
Zum einen eine Form die als Eingabe, Ausgabe und zur Berechnung des Spannbaums verwendet wird. Diese besteht aus drei Arrays, $nodeI$, $nodeJ$ und $weights$, sowie der Anzahl der Knoten ($nodeCount$) und der Kanten ($edgeCount$). Der Graph enthält alle Knoten mit Index $n$ mit $ 0 \le n < nodeCount$ und alle gerichteten Kanten von Knoten $nodeI[i]$ zu $nodeJ[i]$ $\forall 0 \le i < edgeCount$. Das Gewicht der Kante $i$ wird dabei durch $weights[i]$ beschrieben.\\
\\
Zum anderen arbeitet die Dijkstra-Implementation mit einer Darstellung durch Adjazenslisten.


\section{Ergebnisse und Diskussion}

\end{document}
