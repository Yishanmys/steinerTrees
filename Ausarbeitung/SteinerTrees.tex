\documentclass[a4paper,10pt]{article}
\usepackage[utf8]{inputenc}
\usepackage{listings}
\usepackage[square,sort,comma,numbers]{natbib}
\usepackage[german,lined,vlined]{algorithm2e}

%opening
\title{2-Approximation des Pricecollecting-Steintree-Problems}
\author{}

\begin{document}

\maketitle

\begin{abstract}

\end{abstract}

\section{Motivation}
Im Rahmen der Veranstaltung ``Parallele Algorithmen und Datenverarbeitung''

\section{Einleitung}

\section{Verwendeter Algorithmus}
\label{sec:mst}
Der minimimale Steinerbaum wird mithilfe des folgenden Algorithmuses approximiert:\\
Input: Ein Graph und die Liste von Zielknoten (Targets)\\

\section{Implementation}
\subsection{Repräsentation des Graphen}
Zur Repräsentation des Graphen werden zwei Datenstrukturen verwendet.\\
\\
Zum einen eine Form (Kanten-Array-Form) die als Eingabe, Ausgabe und zur Berechnung des Spannbaums verwendet wird. Diese besteht aus drei Arrays, $nodeI$, $nodeJ$ und $weights$, sowie der Anzahl der Knoten ($nodeCount$) und der Kanten ($edgeCount$). Der Graph enthält alle Knoten mit Index $n$ mit $ 0 \le n < nodeCount$ und alle gerichteten Kanten von Knoten $nodeI[i]$ zu $nodeJ[i]$ $\forall 0 \le i < edgeCount$. Das Gewicht der Kante $i$ wird dabei durch $weights[i]$ beschrieben.\\
\\
Zum anderen arbeitet die Dijkstra-Implementation mit einer Darstellung durch Adjazenslisten. Hierbei werden zwei Int-Arrays verwendet. Das erste ($adjacencyList$) ist eine Aneinanderreihung aller Adjazenslisten und hat somit die Länge $edgeCount$. Das zweite ($listStart$) speichert an der Stelle $i$ den Startpunkt der Adjazensliste von Knoten $i$ innerhalb des ersten Arrays. Formal ist die Menge der Kanten also:\\
$\{ (i,j) | 0 \le i < nodeCount, 
j \in \{adjacencyList[n]|listStart[i] \le n < endOf(i)\}$\\ mit $endOf(i) = 
\left\{
	\begin{array}{ll}
		listStart[i+1]  & \mbox{falls } i+1 < nodeCount \\
		edgeCount & \mbox{falls } i+1 \ge nodeCount
	\end{array}
\right.$.

\subsubsection{Konvertierung von Kanten-Array-Form in Adjazenslisten-Form}
Es wurden drei Algorithmen zur Konvertierung implementiert und auf ihre Effiziens getestet.\\
\textbf{ArraySort}\\
Dieser Algorithmus nutzt den bereits implementierten Sortieralgorithmus von Java, wobei der Comperator überschrieben wurde,
sodass die Kanten erst entsprechend dem Startknoten und zweitranging nach dem Zielknoten sortiert werden. Dieser Algorithmus arbeitet
besonders effizient, falls die Eingabe bereits vorsortiert ist. Die allgemeine Effiziensklasse ist jedoch $O(M^2)$ wobei $M$ der Anzahl der Kanten beschreibt.\\
\textbf{CountingSort}\\
Dieser Algorithmus macht sich zu Nutzen, dass es sich beim Sortieren um einen endlichen, abzählbaren Wertebereich ($0 \le n < N$) handelt.
Durch die temporäre überführung in eine Adjazensmatrix wird die Optimale Effiziensklasse $O(N^2)$ mit $N$ der Anzahl der Knoten erreicht.
Dies geschieht jedoch auf Kosten des Speicherbedarfs welcher in jedem Fall $N^2$ beträgt.\\
\textbf{PartlyCountingSort}\\
Da der CountingSort Algorithmus gerade für spärliche Graphen (Diche $< 1\%$) nicht sinnvoll ist, arbeitet dieser Algorithmus etwas dynamischer.
Für jeden Knoten wird eine Ajdazensliste in Form einer ArrayList mit den entsprechenden Folgeknoten gefüllt. Jede dieser Listen wird anschließend sortiert
und anschließend in das adjacencyList-Array geschrieben. Die Effiziensklasse ist $O(N^3)$, jedoch im zu erwartenden Durchschnitt am besten geeignet.

\subsubsection{Iteration über Folgeknoten}
Um über die Folgeknoten eines Knotens $i$ zu iterieren gibt es verschiedene Möglichkeiten.
Die effizienteste nutzt eine for-Schleife über die gesamte Adjazensliste vom Start- bis Endpunkt von $i$:
\begin{lstlisting}
 for(int m=getStartOf(i); m<getEndOf(i); m++) {
    int j = getToNode(m);
    float weight = getWeight(m);
 }
\end{lstlisting}

\subsection{Dijkstra}
Da der Dijkstra-Algorithmus als Rückgabewert die kürzesten Pfade von einem Startknoten $n$ zu allen anderen Knoten des Graphens besitzt,
wurde hierfür eine eigene Klasse entworfen, welche nicht nur als Behälter dient, sondern in Form des Konstruktors auch den Algorithmus implementiert.
Der Konstruktor erwartet somit einen Graphen, repräsentiert als AdjacencyList, und einen Startknoten. Zusätzlich können die Targets des Steinerbaums
angegeben werden, um die Rechenzeit des Dijkstra's zu reduzieren. In diesem Fall ist jedoch nur noch gewährleistet, dass die kürzesten Pfade vom Startknoten
zu allen Targets berechnet wurde und nicht mehr zu allen Knoten.\\
Um den Speicherbedarf des Ergebnisses zu reduzieren, wird nicht für jeden Knoten der gesamte Pfad gespeichert, sondern nur jeweils der Vorgängerknoten.
Der Speicherbedarf sinkt dadurch von $O(N^2)$ auf $O(N)$, jedoch steigt die Zugriffszeit auf den kürzesten Pfad von $O(1)$ auf $O(log(N))$.



\subsection{Minimale Spannbäume}
Die Ausführungen in diesem Unterabschnitt richten sich nach \cite[Kapitel 23]{cormen}.
Der in Abschnitt (\ref{sec:mst}) beschreibene Algorithmus benötigt an zwei Stellen den minimalen Spannbaum eines Graphen $G=(V,E,w)$. Ein Graph
$G_S=(V_S,E_S,w)$ ist ein minimaler Spannbaum für den Graphen G genau dann, wenn $G_S$ eine Teilgraph von G ist, alle Knoten von G enthält und die Summe
seiner Kantengewichte minimal ist. D.h. für $G_S$ muss folgendes gelten: $V_S=V, E_S \subseteq E$ und $\sum_{e \in E_S} w(w)$ muss minimal sein.

Ein möglicher Algorithmus für die Berechnung minimaler Spannbäume ist der Algorithmus von Kruskal. Dieser ist in Algorithmus (\ref{kruskal}) dargestellt.
\begin{algorithm}
  \label{kruskal}
  \caption{Algorithmus von Kruskal}
      \textbf{Input: } $G = (V,E,w)$ \
      $A = \emptyset$ \\
      \For{$v \in V$}
      {
	  MAKE-SET(v)
      }
      Sortiere Kanten aus E in nichtfallender Reihenfolge nach dem Gewicht w \\
      \For{$(u,v) \in E$ in nichtfallender Reihenfolge nach ihren Gewichten}
      {
	  \If{FIND-SET(u) $\neq$ FIND-SET(v)}
	  {
	      $A = A \cup \{ (u,v) \}$ \\
	      UNION(u,v)
	  }
      }
  \hspace{1cm}
\end{algorithm}
\hspace{1cm}
Der Kruskal-Algorithmus startet initial mit einem Wald, in dem jeder Baum aus nur einem Knoten besteht. Dann wird sukzessive über alle Kanten $(u,v)$ des
Eingebegraphen G in nichtfallender Reihenfolge bezüglich des Kantengewichts $w((u,v))$
iteriert. Gehören die Knoten $u$ und $v$ zu verschiedenen Bäumen $B_1$ und $B_2$, wird die Kante
$(u,v)$ zur Menge $A$ hinzugefügt und die Bäume $B_1$ und $B_2$ werden zu einem Baum vereinigt. Am Ende enthält die Menge $A$ die Kanten des
minimalen Spannbaums zum Graphen G.

Für die Verwaltung des Waldes wird eine Datenstruktur disjunkter Mengen mit den Operationen MAKE-SET, FIND-SET und UNION benötigt. Die Operation MAKE-SET
erstellt eine leere Menge, FIND-SET findet die Menge, die das gegebene Element enthält, und UNION vereinigt zwei Mengen. In unserer Implementation
verwenden wir hierfür eine Version, die verkettete Listen für die Repräsentation der Mengen nutzt \cite[Kapitel 21.2]{cormen}. Hier wird für jede
Menge ein repräsentatives Element ausgeweisen (meist das erste Element in der jeweiligen Liste), durch das jede Menge eindeutig repräsentiert wird.
Zusätzlich dazu erhält jedes Element ein Attribut $\mathcal{R}$, mit dem das repräsentative Element der Menge gespeichert wird, zu dem das
jeweilige Element gehört.
Somit benötigt die Operation FIND-SET(E) konstante Laufzeit, da nur das Attribut $\mathcal{R}$ des Elements E abgefragt werden muss.
Nehmen wir für die Operation UNION$(M_1,M_2)$ o.B.d.A. an, dass die Menge $M_2$ der Menge $M_1$ hinzugefügt wird. Um dies zu realisieren muss die
Liste der Menge $M_1$ aktualisiert werden und die Attribute $\mathcal{R}$ der Elemente der Menge $M_2$ müssen auf das repräsentative Element der Menge
$M_1$ gesetzt werden. Hier ist es somit am effizientesten, die kleinere der beiden Mengen $M_1$ und $M_2$ zur größeren hinzuzufügen.
Schließlich kann die Liste der Menge $M_2$ gelöscht werden.

Insgesamt besitzt der Algorithmus von Kruskal eine Laufzeit von $\mathcal{O}(|E|\log(|V|))$.


\section{Ergebnisse und Diskussion}

\bibliography{literatur}
\bibliographystyle{plainnat}

\end{document}
